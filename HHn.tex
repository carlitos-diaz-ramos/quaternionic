\documentclass[12pt, a4paper]{amsart}

\usepackage{amsmath, amsthm, amssymb}

\usepackage[top=37mm, bottom=37mm, left=27mm, right=27mm]{geometry}

\newcommand{\Exp}{\operatorname{Exp}}
\newcommand{\id}{\operatorname{id}}
\newcommand{\g}{\mathfrak}
\newcommand{\ad}{\operatorname{ad}}
\newcommand{\Ad}{\operatorname{Ad}}
\newcommand{\codim}{\operatorname{codim}}
\newcommand{\Span}{\operatorname{span}}
\newcommand{\R}{\mathbb{R}}
\newcommand{\C}{\mathbb{C}}
\renewcommand{\H}{\mathbb{H}}
\renewcommand{\Re}{\operatorname{Re}}
\renewcommand{\Im}{\operatorname{Im}}

\theoremstyle{remark}
\newtheorem*{remark}{Remark}

\begin{document}
\title{Quaternionic projective spaces}

\begin{abstract}
We present the construction and some properties of quaternionic projective spaces.
\end{abstract}

\date{\today}
\maketitle

\section{Quaternions and quaternionic vector spaces}

Quaternions are a normed real division algebra,  
{$\H=\{t+\mathbf{i}x+\mathbf{j}y+\mathbf{k}z:t,x,y,z\in\R\}$}, that can be understood as the vector space $\R^4$. 
Product consists in applying asociative and distributive properties to expressions as before, 
taking into account that $\mathbf{i}^2=\mathbf{j}^2=\mathbf{k}^2=-1$, $\mathbf{i}\mathbf{j}=\mathbf{k}=-\mathbf{j}\mathbf{i}$, and real numbers commute with the symbols 
$\mathbf{i}$, $\mathbf{j}$, $\mathbf{k}$. 
Product is associative, has identity element $1$, and any element has an inverse.

Let $q=t+\mathbf{i}x+\mathbf{j}y+\mathbf{k}z$ be a quaternion. 
The real or scalar part is $\Re{q}=t$ and the imaginary or vector part is 
$\Im{q}=\mathbf{i}x+\mathbf{j}y+\mathbf{k}z$. 
The {conjugate} of $q$ is the quaternion $\overline{q}=t-\mathbf{i}x-\mathbf{j}y-\mathbf{k}z$. 
We have $\overline{q_1 q_2}=\overline{q_2}\,\overline{q_1}$. 
The absolute value or norm of a quaternion is given by
$\lvert q\rvert^2=q\overline{q}=\overline{q}q=t^2+x^2+y^2+z^2$, 
which is the same as the usual norm of $\R^4$. 
An important fact is that norm is multiplicative, that is,
$\lvert pq\rvert=\lvert p\rvert\lvert q\rvert$ for all $p$,~$q\in\H$. 
Moreover, if $q\neq 0$ we have $q^{-1}=\frac{\overline{q}}{\lvert q\rvert^2}$.

If we identify $\Im\H\cong\R^3$, and $\{\mathbf{i},\mathbf{j},\mathbf{k}\}$ is identified with the canonical basis of $\R^3$ then, for any $q_1$, $q_2\in\Im{\H}\cong\R^3$, we have $q_1 q_2=-\langle q_1,q_2\rangle+q_1\times q_2$,
where $\langle\cdot,\cdot\rangle$ is the scalar product of $\R^3\cong\Im\H$, 
and $\cdot\times\cdot$ is the vector product. 

\begin{remark}
We define $\mathsf{Sp}(1)=\{q\in\H:\lvert q\rvert =1\}$.
We consider $\varphi\colon\mathsf{Sp}(1)\to\mathsf{SO}(\Im\H)$, given by $\varphi(q)(x)=qxq^{-1}$. 
The map $\varphi(q)$ is isometric because $\lvert\varphi(q)(x)\rvert=\lvert qxq^{-1}\rvert=\lvert x\rvert$. 
If $\lambda\in\R$ then $\varphi(q)(\lambda)=q\lambda q^{-1}=\lambda$, so $\varphi(q)(\R)=\R$, and thus, $\varphi(q)(\Im\H)=\Im\H$. 
Hence, $\varphi(\mathsf{Sp}(1))\subset\mathsf{O}(\Im\H)$.
Since $\mathsf{Sp}(1)$ is connected, $\det\varphi(\mathsf{Sp}(1))=1$. 
As a consequence, $\varphi(\mathsf{Sp}(1))\subset\mathsf{SO}(\Im\H)$ and this map is well-defined.
Furthermore, $\varphi$ is clearly a Lie group homomorphism because $\mathsf{SO}(\Im\H)$ is a closed subgroup of $\mathsf{GL}(\Im\H)$. 

The Lie algebra $\g{sp}(1)$ of $\mathsf{Sp}(1)$ can be identified with $\Im\H$.
For $u\in\Im\H$ and $t\in\R$ we define $e^{tu}=\cos t+u\sin t$.
Then, 
\[
\varphi_{*1}(u)(v)=\frac{d}{dt}\bigg\vert_0\varphi(e^{tu})v=uv-vu=2(u\times v).
\]
This map is clearly an isomorphism of Lie algebras. 
Therefore, $\varphi$ is a covering of Lie groups.
Moreover, it follows easily that $\ker\varphi={\pm 1}$, and hence, $\mathsf{SO}(3)=\mathsf{Sp}(1)/\mathbb{Z}_2\cong\R\mathsf{P}^3$.
\end{remark}



A quaternion vector space is, in fact, a right $\H$-module.

Let $V$ and $W$ be two quaternion vector spaces. 
A map $L\colon V\to W$ is $\H$-linear if $L(\mathbf{v}q)=L(\mathbf{v})q$ for ach $\mathbf{v}\in V$ and $q\in\H$. Let $\{\mathbf{v}_1,\dots,\mathbf{v}_n\}$ be a basis of $V$ and $\{\mathbf{w}_1,\dots,\mathbf{w}_m\}$ a basis of $W$. 
We may write $L(\mathbf{v}_j)=\sum_i \mathbf{w}_i a_{ij}$, with $a_{ij}\in\H$. 
If $\mathbf{v}=\sum_j \mathbf{v}_j\lambda_j\in V$, $\lambda_j\in\H$ we have 
\[
L(\mathbf{v})
=L\Bigl(\sum_j \mathbf{v}_j\lambda_j\Bigr)
=\sum_j L(\mathbf{v}_j)\lambda_j{}=\sum_j\Bigl(\sum_i \mathbf{w}_i a_{ij}\Bigr)\lambda_j
=\sum_{i}\mathbf{w}_i\Bigl(\sum_j a_{ij}\lambda_j\Bigr),
\]
so the coordinates of $L(\mathbf{v})$ with respect to $\{\mathbf{w}_1,\dots,\mathbf{w}_m\}$ are given by
\[
\begin{pmatrix}
a_{11} & \cdots & a_{1n}\\
\vdots & \ddots & \vdots\\
a_{m1} & \cdots & a_{mn}
\end{pmatrix}
\begin{pmatrix}
\lambda_1\\
\vdots\\
\lambda_n
\end{pmatrix}.
\]
With this convention, composition of $\H$-linear maps corresponds to multiplication of quaternion matrices.

A quaternion product is a map $(\,\cdot\,,\,\cdot\,\,)\colon V\times V\to\H$ such that:
\begin{enumerate}
\item $(\mathbf{x}+\mathbf{y},\mathbf{z})
=(\mathbf{x},\mathbf{z})+(\mathbf{y},\mathbf{z})$,
$(\mathbf{x},\mathbf{y}+\mathbf{z})
=(\mathbf{x},\mathbf{y})+(\mathbf{x},\mathbf{z})$,
\item $(\mathbf{x}q,\mathbf{y})=\overline{q}(\mathbf{x},\mathbf{y})$, $(\mathbf{x},\mathbf{y}q)=(\mathbf{x},\mathbf{y}) q$,
\item $(\mathbf{x},\mathbf{y})=\overline{(\mathbf{y},\mathbf{x})}$,
\item $(\mathbf{x},\mathbf{x})\geq 0$, with equality if and only if $\mathbf{x}=\mathbf{0}$,
\end{enumerate}
for all $\mathbf{x}$, $\mathbf{y}$, $\mathbf{z}\in V$, $q\in H$.

Note that $\langle\mathbf{x},\mathbf{y}\rangle=\Re(\mathbf{x},\mathbf{y})$ defines an inner product in $V$.

In particular, $\H^n$ is quaternion vector space and $(\mathbf{x},\mathbf{y})=\sum_{i=1}^n \overline{x_i}y_i$ defines a quaternion product on $\H^n$. The usual scalar product of $\R^{4n}\cong\H^n$ coincides with $\langle\mathbf{x},\mathbf{y}\rangle=\Re(\mathbf{x},\mathbf{y})$.


We define the general quaternion linear group $\mathsf{GL}(n,\H)$ as the group of invertible quaternion matrices; it is isomorphic with the group of invertible $\H$-linear endomorphisms of $\H^n$. 
The symplectic group is defined as $\mathsf{Sp}(n)=\{A\in\mathsf{GL}(n,\H):A^* A=\id\}$.
This group coincides with the group of quaternion matrices that preserve the quaternion product of $\H^n$.	

\medskip

Let $V$ be a real vector space with inner product $\langle\,\cdot\,,\,\cdot\,\rangle$.
A $3$-dimensional subspace $\g{q}$ of $\g{so}(V)$ is called a linear quaternionic structure if there are elements $J_1$, $J_2$, $J_3\in\g{q}$ such that $J_i^2=-1$ and $J_iJ_{i+1}=J_{i+2}$ (indices modulo 3).
Then, $\g{q}$ is a subalgebra of $\g{so}(V)$ isomorphic to $\g{sp}(1)$.

If $V$ is a quaternionic vector space with a quaternionic product as above, we define $J_1(\mathbf{x})=-\mathbf{x}\mathbf{i}$, $J_2(\mathbf{x})=-\mathbf{x}\mathbf{j}$, $J_3(\mathbf{x})=-\mathbf{x}\mathbf{k}$.
Then, $\g{q}=\R J_1\oplus\R J_2\oplus\R J_3$ is a linear quaternionic structure on $V$.

Conversely, if $\g{q}$ is a linear quaternionic structure on $V$, then we can define a right quaternionic vector space structure on $V$ as $\mathbf{v}(t+x\mathbf{i}+y\mathbf{j}+z\mathbf{k})=t\mathbf{v}+xJ_1(\mathbf{v})+yJ_2(\mathbf{v})+zJ_3(\mathbf{v})$.

\section{Projective spaces}

Let $V$ be a quaternion vector space of quaternionic dimension $n+1$ endowed with a quaternion product $(\,\cdot\,,\,\cdot\,\,)$, and $\langle\,\cdot\,,\,\cdot\,\,\rangle=\Re(\,\cdot\,,\,\cdot\,\,)$.
Then $V$ is isomorphic with $\H^{n+1}$ with the usual quaternion product.

We consider the sphere $\mathsf{S}^{4n+3}=\mathsf{S}(V)=\{\mathbf{x}\in V:\langle\mathbf{x},\mathbf{x}\rangle=1\}$.
We define the right action $\mathsf{S}(V)\times\mathsf{Sp}(1)\to\mathsf{S}(V)$ by $\mathbf{x}\cdot \lambda=\mathbf{x}\lambda$.
This action is obviously proper and free. 
Hence, the orbit space $\mathsf{P}(V)=\mathsf{S}(V)/\mathsf{Sp}(1)$ is a smooth manifold of real dimension $4n$, that is called the quaternion projective space of $V$.
If $V=\H^{n+1}$, then we denote this space by $\H \mathsf{P}^n$.

We denote by $\pi\colon\mathsf{S}(V)\to\mathsf{P}(V)$ the quotient map.
This map is a submersion that is called the Hopf map.
If $\mathbf{x}\in\mathsf{S}(V)$, then we can identify $T_\mathbf{x}\mathsf{S}(V)=(\R\mathbf{x})^\perp=V\ominus\R\mathbf{x}$, the orthogonal complement of $\mathbf{x}$ with respect to $\langle\,\cdot\,,\,\cdot\,\,\rangle$.
Thus, $\ker\pi_{*\mathbf{x}}=\mathbf{x}\Im\H$, and we have an orthogonal decomposition $T_\mathbf{x}\mathsf{S}(V)=\mathbf{x}\Im\H\oplus(\H\mathbf{x})^\perp$.
We endow $\mathsf{P}(V)$ with a Riemannian metric that makes $\pi$ a Riemannian submersion.
Thus, $T_{\pi(\mathbf{x})}\mathsf{P}(V)$ becomes isometric to $(\H\mathbf{x})^\perp$ through $\pi_{*\mathbf{x}}$.

Let $v\in T_{p}\mathsf{P}(V)$.
For $\mathbf{x}\in\pi^{-1}(p)$ we define the horizontal lift $v^L_\mathbf{x}$ of $v$ as the unique vector in $(\H\mathbf{x})^\perp$ such that $\pi_{*\mathbf{x}}(v^L_\mathbf{x})=v$.
The fact that $\pi$ is a Riemannian submersion implies that the metric on $\mathsf{P}(V)$ is defined as 
$\langle v,w\rangle=\langle v^L,w^L\rangle$ for each $v$, $w\in T_p\mathsf{P}(V)$.
We see that this definition does not depend on the point $\mathbf{x}\in\pi^{-1}(p)$ used to define the lift.
Firstly, $v^L_{\mathbf{x}\lambda}=v^L_\mathbf{x}\lambda$, for $\lambda\in\mathsf{Sp}(1)$. Then,
\[
\langle v^L_{\mathbf{x}\lambda},w^L_{\mathbf{x}\lambda}\rangle=
\langle v^L_{\mathbf{x}}\lambda,w^L_{\mathbf{x}}\lambda\rangle=
\Re\bar{\lambda}(v^L_\mathbf{x},w^L_\mathbf{x})\lambda=
\Re\lvert\lambda\rvert^2(v^L_\mathbf{x},w^L_\mathbf{x})=
\langle v^L_{\mathbf{x}},w^L_{\mathbf{x}}\rangle.
\]

In $V$ we can consider the action of $\mathsf{Sp}(V)\cong\mathsf{Sp}(n+1)$ on the left as
$A\cdot\mathbf{x}=A\mathbf{x}$, where matrix multiplication is $\H$-linear on the right.
This action leaves the sphere $\mathsf{S}(V)$ invariant and descents to $\mathsf{P}(V)$ as $A\cdot\pi(\mathbf{x})=\pi(A\cdot\mathbf{x})$.

The action of $\mathsf{Sp}(n+1)$ on $\mathsf{P}(V)$ is transitive and its effectivity kernel is $\{\pm 1\}$.

%We calculate the effectivity kernel of the $\mathsf{Sp}(V)$ action on $\mathsf{P}(V)$.
%Assume $A\cdot\pi(\mathbf{x})=\pi(\mathbf{x})$ for all $\mathbf{x}\in V$. Then, there exists $\lambda\colon V\to\mathsf{Sp}(1)$ such that $A\mathbf{x}=\mathbf{x}\lambda(\mathbf{x})$. If $\{\mathbf{e}_1,\dots,\mathbf{e}_{n+1}\}$ is a quaternion orthonormal basis of $V$, then $A\mathbf{e}_i=\mathbf{e}_i\lambda(\mathbf{e}_i)$ implies $A=\operatorname{diag}(\lambda(\mathbf{e}_1),\dots\lambda(\mathbf{e}_{n+1}))$.
\end{document}