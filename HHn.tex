\documentclass[12pt, a4paper]{amsart}

\usepackage{amsmath, amsthm, amssymb}
\usepackage[top=3cm,right=2.5cm,bottom=3cm,left=2.5cm]{geometry}

\usepackage{lmodern}
\usepackage{pxfonts}
\usepackage{xcolor}
\usepackage[]{hyperref}
\hypersetup{
	colorlinks,
	linkcolor={red!40!black},
	citecolor={green!40!black},
	urlcolor={blue!40!black},
}
\usepackage[icon=Note,color=yellow,open=true]{pdfcomment}

\newcommand{\Exp}{\operatorname{Exp}}
\newcommand{\id}{\operatorname{id}}
\newcommand{\g}{\mathfrak}
\newcommand{\ad}{\operatorname{ad}}
\newcommand{\Ad}{\operatorname{Ad}}
\newcommand{\tr}{\operatorname{tr}}
\newcommand{\Span}{\operatorname{span}}
\newcommand{\R}{\mathbb{R}}
\newcommand{\C}{\mathbb{C}}
\renewcommand{\H}{\mathbb{H}}
\renewcommand{\Re}{\operatorname{Re}}
\renewcommand{\Im}{\operatorname{Im}}
\newcommand{\Sp}{\mathsf{Sp}}

\newtheorem{theorem}{Theorem}[section]
\newtheorem{proposition}[theorem]{Proposition}
\newtheorem{lemma}[theorem]{Lemma}
\theoremstyle{remark}
\newtheorem*{remark}{Remark}

\begin{document}
\title{Quaternionic projective and hyperbolic spaces}

\begin{abstract}
We present the construction and some properties of quaternionic projective and hyperbolic spaces.
\end{abstract}

\date{\today}
\maketitle

\section{Quaternions}

Quaternions are a normed real division algebra,  
{$\H=\{t+\mathbf{i}x+\mathbf{j}y+\mathbf{k}z:t,x,y,z\in\R\}$}, that can be understood as the vector space $\R^4$. 
Product consists in applying asociative and distributive properties to expressions as before, 
taking into account that $\mathbf{i}^2=\mathbf{j}^2=\mathbf{k}^2=-1$, $\mathbf{i}\mathbf{j}=\mathbf{k}=-\mathbf{j}\mathbf{i}$, and real numbers commute with the symbols 
$\mathbf{i}$, $\mathbf{j}$, $\mathbf{k}$. 
Product is associative, has identity element $1$, and any nonzero element has an inverse.

Let $q=t+\mathbf{i}x+\mathbf{j}y+\mathbf{k}z$ be a quaternion. 
The real or scalar part is $\Re{q}=t$ and the imaginary or vector part is 
$\Im{q}=\mathbf{i}x+\mathbf{j}y+\mathbf{k}z$. 
The {conjugate} of $q$ is the quaternion $\overline{q}=t-\mathbf{i}x-\mathbf{j}y-\mathbf{k}z$. 
We have $\overline{q_1 q_2}=\overline{q_2}\,\overline{q_1}$. 
The absolute value or norm of a quaternion is given by
$\lvert q\rvert^2=q\overline{q}=\overline{q}q=t^2+x^2+y^2+z^2$, 
which is the same as the usual norm of $\R^4$. 
An important fact is that norm is multiplicative, that is,
$\lvert pq\rvert=\lvert p\rvert\lvert q\rvert$ for all $p$,~$q\in\H$. 
Moreover, if $q\neq 0$ we have $q^{-1}=\frac{\overline{q}}{\lvert q\rvert^2}$.

The quaternion product is determined by the following rules:
\begin{itemize}
\item $\Re{(q_1 q_2)}=(\Re{q_1})(\Re{q_2})-\langle \Im{q_1},\Im{q_2}\rangle$,
\item $\Im{(q_1 q_2)}=(\Re{q_1})(\Im{q_2})+(\Re{q_2})(\Im{q_1})+\Im{q_1}\times\Im{q_2}$,
\end{itemize}
where $\langle\cdot,\cdot\rangle$ is the scalar product of $\R^3\cong\Im\H$, 
and $\cdot\times\cdot$ is the vector product. 
From the equations above, if $\lambda\in\H$ is such that $q\lambda=\lambda q$ for any $q\in\H$, then $\lambda\in\R$, that is $Z(\H)=\R$.
Let $q\in\H$; then, $q\in\Im\H$ if and only if $q^2\leq 0$.

With the identification $\Im\H\cong\R^3$, and if $\{\mathbf{i},\mathbf{j},\mathbf{k}\}$ is identified with the canonical basis of $\R^3$ then, for any $q_1$, $q_2\in\Im{\H}\cong\R^3$, we have 
\[
\begin{aligned}
q_1 q_2
&{}=-\langle q_1,q_2\rangle+q_1\times q_2,
&q_1\times q_2
&{}=\frac{1}{2}(q_1q_2-q_2q_1).
\end{aligned}
\]
For $q_1$, $q_2$, $q_3\in\Im\H$ we define the scalar triple product of $q_1$, $q_2$, and $q_3$ as $\Re(q_1q_2q_3)$.
Then, we have
\[
\Re(q_1q_2q_3)=\langle q_1,q_2\times q_3\rangle=\det(q_1\vert q_2\vert q_3).
\]

\begin{remark}
We define $\Sp(1)=\{q\in\H:\lvert q\rvert =1\}$.
This is a Lie subgroup of $\H\setminus\{0\}$.
We consider $\varphi\colon\Sp(1)\to\mathsf{SO}(\Im\H)$, given by $\varphi(q)(x)=qxq^{-1}$. 
The map $\varphi(q)$ is isometric because $\lvert\varphi(q)(x)\rvert=\lvert qxq^{-1}\rvert=\lvert x\rvert$. 
If $\lambda\in\R$ then $\varphi(q)(\lambda)=q\lambda q^{-1}=\lambda$, so $\varphi(q)(\R)=\R$, and thus, $\varphi(q)(\Im\H)=\Im\H$. 
Hence, $\varphi(\Sp(1))\subset\mathsf{O}(\Im\H)$.
Since $\Sp(1)$ is connected, $\det\varphi(\Sp(1))=1$. 
As a consequence, $\varphi(\Sp(1))\subset\mathsf{SO}(\Im\H)$ and this map is well-defined.
Furthermore, $\varphi$ is clearly a Lie group homomorphism because $\mathsf{SO}(\Im\H)$ is a closed subgroup of $\mathsf{GL}(\Im\H)$. 

The Lie algebra $\g{sp}(1)$ of $\Sp(1)$ can be identified with $\Im\H$.
For $u\in\Im\H$ and $t\in\R$ we define $e^{tu}=\cos t+u\sin t$.
Then, 
\[
\varphi_{*1}(u)(v)=\frac{d}{dt}\bigg\vert_0\varphi(e^{tu})v=uv-vu=2(u\times v).
\]
This map is clearly an isomorphism of Lie algebras. 
Therefore, $\varphi$ is a covering of Lie groups.
Moreover, it follows easily that $\ker\varphi={\pm 1}$, and hence, $\mathsf{SO}(3)=\Sp(1)/\mathbb{Z}_2\cong\R\mathsf{P}^3$.
\end{remark}

\section{Quaternionic vector spaces}\label{sec:vector-spaces}

A quaternionic vector space is, in fact, a right $\H$-module.
In particular, $\H^n$ is quaternion vector space.

Let $V$ and $W$ be two quaternion vector spaces. 
A map $L\colon V\to W$ is $\H$-linear if $L(\mathbf{v}q)=L(\mathbf{v})q$ for each $\mathbf{v}\in V$ and $q\in\H$. Let $\{\mathbf{v}_1,\dots,\mathbf{v}_n\}$ be a basis of $V$ and $\{\mathbf{w}_1,\dots,\mathbf{w}_m\}$ a basis of $W$. 
We may write $L(\mathbf{v}_j)=\sum_i \mathbf{w}_i a_{ij}$, with $a_{ij}\in\H$. 
If $\mathbf{v}=\sum_j \mathbf{v}_j\lambda_j\in V$, $\lambda_j\in\H$, we have 
\[
L(\mathbf{v})
=L\Bigl(\sum_j \mathbf{v}_j\lambda_j\Bigr)
=\sum_j L(\mathbf{v}_j)\lambda_j{}=\sum_j\Bigl(\sum_i \mathbf{w}_i a_{ij}\Bigr)\lambda_j
=\sum_{i}\mathbf{w}_i\Bigl(\sum_j a_{ij}\lambda_j\Bigr),
\]
so the coordinates of $L(\mathbf{v})$ with respect to $\{\mathbf{w}_1,\dots,\mathbf{w}_m\}$ are given by
\[
\begin{pmatrix}
a_{11} & \cdots & a_{1n}\\
\vdots & \ddots & \vdots\\
a_{m1} & \cdots & a_{mn}
\end{pmatrix}
\begin{pmatrix}
\lambda_1\\
\vdots\\
\lambda_n
\end{pmatrix}.
\]
With this convention, composition of $\H$-linear maps corresponds to multiplication of quaternion matrices.

We define the general quaternion linear group $\mathsf{GL}(n,\H)$ as the group of invertible quaternion matrices; it is isomorphic to the group of invertible $\H$-linear endomorphisms of $\H^n$. 

We can identify $\H^n$ with $\C^n\oplus\C^n$ by means of the isomorphism
\[
\C^n\oplus\C^n\to\H^n,\quad (\mathbf{x},\mathbf{y})\mapsto \mathbf{x}+\mathbf{j}\mathbf{y}.
\]
Thus, given a quaternionic matrix $M$, we may write $M=A+\mathbf{j}B$.
Note that
\[
\begin{aligned}
(A+\mathbf{j}B)(\mathbf{x}+\mathbf{j}\mathbf{y})
&{}=A\mathbf{x}+A\mathbf{j}\mathbf{y}
+\mathbf{j}B\mathbf{x}+\mathbf{j}B\mathbf{j}\mathbf{y}\\
&{}=(A\mathbf{x}-\bar{B}\mathbf{y})
+\mathbf{j}(\bar{A}\mathbf{y}+B\mathbf{x}).
\end{aligned}
\]
Hence, quaternionic matrices can be seen as matrices of $\C^{2n}$ of the form
\[
A+\mathbf{j}B\mapsto
\begin{pmatrix}
A & -\overline{B}\\
B & \overline{A}
\end{pmatrix}.
\]

Using the previous isomorphism, multiplication on the right by $\mathbf{j}$ corresponds to the anti-linear complex endomorphism $\hat{J}(\mathbf{x},\mathbf{y})=(-\overline{\mathbf{y}},\overline{\mathbf{x}})$.
As a consequence, a complex linear endomorphism of $\C^n\oplus\C^n$ corresponds to an $\H$-linear endomorphism if and only if it commutes with $\hat{J}$. 
(Indeed, because for all $\mathbf{v}\in\C^n\oplus\C^n$ we have $M\hat{J}\mathbf{v}=M(\mathbf{v}\mathbf{j})=(M\mathbf{v})\mathbf{j}=\hat{J}M\mathbf{v}$.)

We also consider
\[
{J}=
\begin{pmatrix}
\mathbf{0} & -\id_n\\
\id_n & \mathbf{0}
\end{pmatrix}.
\]
Thus, $\hat{J}\mathbf{v}=J\bar{\mathbf{v}}$, $\mathbf{v}\in\C^n\oplus\C^n$.
In terms of matrices $M\in\mathcal{M}_{2n\times 2n}(\C)$ corresponds to an element of $\mathcal{M}_{n\times n}(\H)$ if and only if $JM=\bar{M}J$
(simply apply this definition and conjugate $M\hat{J}=\hat{J}M$).

Hence, with these isomorphisms, 
\[
\begin{aligned}
\mathsf{GL}(n,\H)
&{}=\{M\in\mathsf{GL}(2n,\C):M\hat{J}=\hat{J}M\}\\
&{}=\{M\in\mathsf{GL}(2n,\C):\overline{M}J=JM\}.
\end{aligned}
\]
\medskip

Let $V$ be a real vector space with inner product $\langle\,\cdot\,,\,\cdot\,\rangle$.
A $3$-dimensional subspace $\g{q}$ of $\g{so}(V)$ is called a linear quaternionic structure if there are elements $J_1$, $J_2$, $J_3\in\g{q}$ such that $J_i^2=-1$ and $J_iJ_{i+1}=J_{i+2}$ (indices modulo 3).
Then, $\g{q}$ is a subalgebra of $\g{so}(V)$ isomorphic to $\g{sp}(1)$.
The set $\{J_1,J_2,J_3\}$ is called a canonical basis for $\g{q}$.

If $V$ is a quaternionic vector space with a quaternionic product as above, we define $J_1(\mathbf{x})=-\mathbf{x}\mathbf{i}$, $J_2(\mathbf{x})=-\mathbf{x}\mathbf{j}$, $J_3(\mathbf{x})=-\mathbf{x}\mathbf{k}$.
Then, $\g{q}=\R J_1\oplus\R J_2\oplus\R J_3$ is a linear quaternionic structure on $V$.

\begin{proof}
For example,
\[
J_1 J_2(\mathbf{v})
=J_1(-\mathbf{v}\mathbf{j})
=\mathbf{v}\mathbf{j}\mathbf{i}
=-\mathbf{v}\mathbf{k}
=J_3(\mathbf{v}).
\]
We also show as an example,
\begin{align*}
\langle J_1\mathbf{v},J_1\mathbf{w}\rangle
&{}=\langle -\mathbf{v}\mathbf{i},-\mathbf{v}\mathbf{i}\rangle
=\Re (\mathbf{v}\mathbf{i},\mathbf{v}\mathbf{i})\\
&{}=\Re -\mathbf{i} (\mathbf{v},\mathbf{w}) \mathbf{i}
=\Re (\mathbf{v},\mathbf{w}) (-\mathbf{i}\mathbf{i})
=\Re (\mathbf{v},\mathbf{w})
=\langle \mathbf{v},\mathbf{w}\rangle.\qedhere
\end{align*}
\end{proof}

Conversely, if $\g{q}$ is a linear quaternionic structure on $V$, then we can define a right quaternionic vector space structure on $V$ as $\mathbf{v}(t+x\mathbf{i}+y\mathbf{j}+z\mathbf{k})=t\mathbf{v}+xJ_1(\mathbf{v})+yJ_2(\mathbf{v})+zJ_3(\mathbf{v})$.

\begin{proof}
For example, we check
\[
\begin{aligned}
\mathbf{v}\mathbf{i}\mathbf{j}
=J_1(\mathbf{v})\mathbf{j}
=J_2 J_1(\mathbf{v})
=-J_1 J_2(\mathbf{v})
=-J_3(\mathbf{v})
=\mathbf{v}\mathbf{k}.
\end{aligned} \qedhere
\]
\end{proof}
\medskip

A quaternion product is a map $(\,\cdot\,,\,\cdot\,\,)\colon V\times V\to\H$ such that:
\begin{enumerate}
\item $(\mathbf{x}+\mathbf{y},\mathbf{z})
=(\mathbf{x},\mathbf{z})+(\mathbf{y},\mathbf{z})$,
$(\mathbf{x},\mathbf{y}+\mathbf{z})
=(\mathbf{x},\mathbf{y})+(\mathbf{x},\mathbf{z})$,
\item $(\mathbf{x}q,\mathbf{y})=\overline{q}(\mathbf{x},\mathbf{y})$, $(\mathbf{x},\mathbf{y}q)=(\mathbf{x},\mathbf{y}) q$,
\item $(\mathbf{x},\mathbf{y})=\overline{(\mathbf{y},\mathbf{x})}$,
\item $(\mathbf{x},\mathbf{x})\geq 0$, with equality if and only if $\mathbf{x}=\mathbf{0}$,
\end{enumerate}
for all $\mathbf{x}$, $\mathbf{y}$, $\mathbf{z}\in V$, $q\in H$.

Note that $\langle\mathbf{x},\mathbf{y}\rangle=\Re(\mathbf{x},\mathbf{y})$ defines an inner product in $V$.

The formula $(\mathbf{x},\mathbf{y})=\sum_{i=1}^{n+1} \overline{x_i}y_i$ defines a quaternion product on $\H^{n+1}$. The usual scalar product of $\R^{4n+4}\cong\H^{n+1}$ coincides with $\langle\mathbf{x},\mathbf{y}\rangle=\Re(\mathbf{x},\mathbf{y})$.

The symplectic group is defined as $\Sp(n+1)=\{A\in\mathsf{GL}(n+1,\H):A^* A=\id\}$.
This group coincides with the group of quaternion matrices that preserve the quaternion product of $\H^{n+1}$.	
We have $\dim\Sp(n+1)=2(n+1)^2+n+1$.

We are also interested in indefinite quaternion products, where the last property of a quaternion product is substituted by non-degeneracy.
In particular, we are interested in $(\mathbf{x},\mathbf{y})=-\overline{x_1}y_1+\sum_{i=2}^{n+1} \overline{x_i}y_i$ on $\H^{n+1}$.
The quaternion vector space $\H^{n+1}$ with this indefinite quaternion product is denoted by $\H^{1,n}$.
We define $\Sp(1,n)=\{A\in\mathsf{GL}(n+1,\H):A^*I_{1,n}A=I_{1,n}\}$, where $I_{1,n}=\operatorname{diag}(-1,1,\dots,1)$.
This group coincides with the group of quaternion matrices that preserve the indefinite quaternion product of $\H^{1,n}$.	


\section{Quaternionic projective spaces}

Let $V$ be a quaternion vector space of quaternionic dimension $n+1$ endowed with a quaternion product $(\,\cdot\,,\,\cdot\,\,)$, and $\langle\,\cdot\,,\,\cdot\,\,\rangle=\Re(\,\cdot\,,\,\cdot\,\,)$.
Then $V$ is isomorphic with $\H^{n+1}$ with the usual quaternion product.

We consider the sphere $\mathsf{S}^{4n+3}=\mathsf{S}(V)=\{{x}\in V:\langle {x},{x}\rangle=1\}$.
We define the right action $\mathsf{S}(V)\times\Sp(1)\to\mathsf{S}(V)$ by ${x}\cdot \lambda={x}\lambda$.
This action is obviously proper and free. 
Hence, the orbit space 
\[
\mathsf{P}(V)=\mathsf{S}(V)/\Sp(1)
\]
is a smooth manifold of real dimension $4n$, that is called the quaternion projective space of $V$.
If $V=\H^{n+1}$, then we denote this space by $\H \mathsf{P}^n$.

We denote by $\pi\colon\mathsf{S}(V)\to\mathsf{P}(V)$ the quotient map.
This map is a submersion that is called the Hopf map.
If ${x}\in\mathsf{S}(V)$, then we can identify $T_{x}\mathsf{S}(V)=(\R{x})^\perp=V\ominus\R{x}$, the orthogonal complement of ${x}$ with respect to $\langle\,\cdot\,,\,\cdot\,\,\rangle$.
Thus, $\ker\pi_{*{x}}={x}\Im\H$, and we have an orthogonal decomposition 
\[
T_{x}\mathsf{S}(V)=({x}\Im\H)\oplus(\H{x})^\perp.
\]
We endow $\mathsf{P}(V)$ with a Riemannian metric that makes $\pi$ a Riemannian submersion.
Thus, $T_{\pi({x})}\mathsf{P}(V)$ becomes isometric to $(\H{x})^\perp$ through $\pi_{*{x}}$.
A vector in $(\H x)^\perp$ is called horizontal, and a vector in $\ker\pi_{*x}$ is called vertical, as it is common for Riemannian submersions~\cite{ONeill}.

Let $v\in T_{p}\mathsf{P}(V)$.
For ${x}\in\pi^{-1}(p)$ we define the horizontal lift $v^L_{x}$ of $v$ as the unique vector $v_x^L\in(\H{x})^\perp$ such that $\pi_{*{x}}(v^L_{x})=v$.
Note that, with the usual identifications, we have $v_{x\lambda}^L=v_x^L\lambda$, with $\lambda\in\Sp(1)$, since $(x\Im\H)^\perp$ is $\H$-invariant and $\pi_{x\lambda}(v_x^L\lambda)=\pi_{*x}(v_x^L)$.
\medskip

In order to give $\mathsf{P}(V)$ a structure of a Riemannian manifold, we require $\pi$ to be a Riemannian submersion.
This implies that the metric on $\mathsf{P}(V)$ is defined as 
\[
\langle v,w\rangle_{\mathsf{P}(V)}=\langle v^L,w^L\rangle_{\mathsf{S}(V)}
\] 
for each $v$, $w\in T_p\mathsf{P}(V)$.
We will omit subindices whenever the space where we are considering the Riemannian metric is clear.
We see that this definition does not depend on the point ${x}\in\pi^{-1}(p)$ used to define the lift.
Indeed,
\[
\langle v^L_{{x}\lambda},w^L_{{x}\lambda}\rangle=
\langle v^L_{{x}}\lambda,w^L_{{x}}\lambda\rangle=
\Re\bar{\lambda}(v^L_{x},w^L_{x})\lambda=
\Re\lvert\lambda\rvert^2(v^L_{x},w^L_{x})=
\langle v^L_{x},w^L_{x}\rangle.
\]

Since $\pi\colon\mathsf{S}(V)\to\mathsf{P}(V)$ is a Riemannian submersion, the Levi-Civita connection $\nabla$ of $\mathsf{P}(V)$ is determined by~\cite{ONeill}
\[
(\nabla_X Y)_{\pi(x)}=\pi_{*x}(\tilde{\nabla}_X Y),
\]
where $\tilde{\nabla}$ is the Levi-Civita connection of $\mathsf{S}(V)$.
\medskip

Now we define a quaternion K\"ahler structure on $\mathsf{P}(V)$.

Let $\sigma\colon U\to\mathsf{S}(V)$ be a section of $\pi$, that is, a smooth map such that $\pi\circ\sigma=\id$ in the open set $U$ of $\mathsf{P}(V)$.
We define, for $X\in\Gamma(T\mathsf{P}(V))$,
\[
\begin{aligned}
J_1(X_p)&{}=\pi_{*\sigma(p)}(-X_{\sigma(p)}^L\mathbf{i}),
&J_2(X_p)&{}=\pi_{*\sigma(p)}(-X_{\sigma(p)}^L\mathbf{j}),
&J_3(X_p)&{}=\pi_{*\sigma(p)}(-X_{\sigma(p)}^L\mathbf{k}),
\end{aligned}
\]
for $p\in U$.
The same definition can be made for $v\in T_p\mathsf{P}(V)$.
It is important to note, however, that this definition depends on the choice of section $\sigma$.

\begin{lemma}
The above definition determines a quaternionic K\"ahler structure on $\mathsf{P}(V)$.
\end{lemma}

\begin{proof}
In order to shorten the notation we define $\mathcal{V}=\ker\pi_{*\sigma(p)}=\sigma(p)\Im\H$, and $\mathcal{V}^\perp$ its quaternionic orthogonal complement.
Note that ${\mathcal{V}}^\perp$ is $\H$-invariant, a fact that we will use often in this proof.

First we check that $J_i^2=-\id$.
For example, 
\[
\begin{aligned}
J_1^2(X_p)
=J_1(\pi_{*\sigma(p)}(-X_{\sigma(p)}^L\mathbf{i}))
=\pi_{*\sigma(p)}(X_{\sigma(p)}^L\mathbf{i}\mathbf{i})
=-X_p,
\end{aligned}
\]
using that $\mathcal{V}^\perp$ is $\H$-invariant, and thus, multiplication by $\mathbf{i}$ maps horizontal vectors to horizontal vectors.

Similarly,
\[
\begin{aligned}
J_1 J_2(X_p)
=J_1(\pi_{*\sigma(p)}(-X_{\sigma(p)}^L\mathbf{j}))
=\pi_{*\sigma(p)}(X_{\sigma(p)}^L\mathbf{j}\mathbf{i})
=\pi_{*\sigma(p)}(-X_{\sigma(p)}^L\mathbf{k})
=J_3(X_p).
\end{aligned}
\]

Since $\pi$ is a Riemannian submersion,
\[
\begin{aligned}
\langle J_1(X_p),J_1(Y_p)\rangle
=\langle \pi_{*\sigma(p)}(-X_{\sigma(p)}^L\mathbf{i}),\pi_{*\sigma(p)}(-Y_{\sigma(p)}^L\mathbf{i})\rangle
=\langle X_{\sigma(p)}^L\mathbf{i}, Y_{\sigma(p)}^L\mathbf{i}\rangle
=\langle X_{\sigma(p)}^L, Y_{\sigma(p)}^L\rangle
=\langle X_p,Y_p\rangle.
\end{aligned}
\]

Now we show that the space $\g{q}$ generated by $\{J_1,J_2,J_3\}$ is invariant with respect to the Levi-Civita connection, that is, $\nabla_X\g{q}\subset\g{q}$.
To that end we calculate $(\nabla_X J_1)_p Y$, $p\in \mathsf{P}(V)$, $X$, $Y\in\Gamma(TU)$.

Let $\gamma\colon I\to U$ be a smooth curve such that $\gamma(0)=p$, $\gamma'(0)=X_p$.
We denote by $\tilde{\gamma}$ its horizontal lift to $\sigma(U)\subset\mathsf{S}(V)$ through $\sigma(p)$.
Thus, $\tilde{\gamma}(0)=\sigma(p)$, $\tilde{\gamma}'(0)=X_p^L$.
We write $\tilde{\gamma}(t)=\sigma(\gamma(t))\lambda(t)$, where $\lambda$ is a smooth function in $\Sp(1)$.
Since $\sigma(p)=\tilde{\gamma}(0)=\sigma(p)\lambda(0)$ we get $\lambda(0)=1$ and $\lambda'(0)\in T_1\Sp(1)=\Im\H$.

Recall that $\sigma(p)^\perp=T_{\sigma(p)}\mathsf{S}(V)$. In what follows subspaces of $\sigma(p)^\perp$ written as subindices mean orthogonal projection to that subspace.
Recall that $\tilde{\nabla}$ is the Levi-Civita connection of $\mathsf{S}(V)$.
We also denote by $D$ the usual connection of $\H^{n+1}\cong\R^{4n+4}$.

On the one hand,
\[
\begin{aligned}
\nabla_{X_p}(J_1 Y)
&{}=\pi_{*\sigma(p)}\Bigl(\tilde{\nabla}_{X_{\sigma(p)}^L}(J_1 Y)^L\Bigr)
=\pi_{*\sigma(p)}\Bigl(\Bigl(D_{X_{\sigma(p)}^L}(J_1 Y)^L\Bigr)_{\sigma(p)^\perp}\Bigr)\\
&{}=\pi_{*\sigma(p)}\left(
\left(\frac{d}{dt}\Big\vert_0 (J_1 Y)_{\tilde{\gamma}(t)}^L\right)_{\sigma(p)^\perp}\right)
=\pi_{*\sigma(p)}\left(
\left(\frac{d}{dt}\Big\vert_0 (J_1 Y)_{\sigma(\gamma(t))\lambda(t)}^L\right)_{\sigma(p)^\perp}\right)\\
&{}=\pi_{*\sigma(p)}\left(
\left(\frac{d}{dt}\Big\vert_0 (J_1 Y)_{\sigma(\gamma(t))}^L\lambda(t)\right)_{\sigma(p)^\perp}\right)
=\pi_{*\sigma(p)}\left(
\left(-\frac{d}{dt}\Big\vert_0 Y_{\sigma(\gamma(t))}^L\mathbf{i}\lambda(t)\right)_{\sigma(p)^\perp}\right)\\
&{}=\pi_{*\sigma(p)}\left(
-\left(\frac{d(Y^L\circ\sigma\circ\gamma)}{dt}(0)\,\mathbf{i}\lambda(0) +Y_{\sigma(\gamma(0))}^L\mathbf{i}\lambda'(0)\right)_{\sigma(p)^\perp}\right)\\
&{}=\pi_{*\sigma(p)}\left(
-\left(\frac{d(Y^L\circ\sigma\circ\gamma)}{dt}(0)\,\mathbf{i}\right)_{\mathcal{V}^\perp} -Y_{\sigma(p)}^L\mathbf{i}\lambda'(0)\right).
\end{aligned}
\]
On the other hand,
\[
\begin{aligned}
J_1(\nabla_{X_p}Y)
&{}=\pi_{*\sigma(p)}\Bigl(-(\nabla_X Y)_{\sigma(p)}^L\mathbf{i}\Bigr)
=\pi_{*\sigma(p)}\Bigl(
-\Bigl(\pi_{*\sigma(p)}(\tilde{\nabla}_{X_{\sigma(p)}}^L Y^L)\Bigr)_{\sigma(p)}^L\mathbf{i}\Bigr)\\
&{}=\pi_{*\sigma(p)}\Bigl(
-\Bigl(\tilde{\nabla}_{X_{\sigma(p)}}^L Y^L\Bigr)_{\mathcal{V}^\perp}\mathbf{i}\Bigr)
=\pi_{*\sigma(p)}\left(
-\left(\frac{d}{dt}\Big\vert_0 Y_{\tilde{\gamma}(t)}^L\right)_{\mathcal{V}^\perp}\mathbf{i}\right)\\
&{}=\pi_{*\sigma(p)}\left(
-\left(\frac{d}{dt}\Big\vert_0  Y_{\sigma(\gamma(t))\lambda(t)}^L\right)_{\mathcal{V}^\perp}\mathbf{i}\right)
=\pi_{*\sigma(p)}\left(
-\left(\frac{d}{dt}\Big\vert_0  Y_{\sigma(\gamma(t))}^L\lambda(t)\right)_{\mathcal{V}^\perp}\mathbf{i}\right)\\
&{}=\pi_{*\sigma(p)}\left(
-\left(\frac{d(Y^L\circ\sigma\circ\gamma)}{dt}(0)\lambda(0) +Y_{\sigma(\gamma(0))}^L\lambda'(0)\right)_{\mathcal{V}^\perp}\mathbf{i}\right)\\
&{}=\pi_{*\sigma(p)}\left(
-\left(\frac{d(Y^L\circ\sigma\circ\gamma)}{dt}(0)\right)_{\mathcal{V}^\perp}\mathbf{i} -Y_{\sigma(p)}^L\lambda'(0)\mathbf{i}\right).
\end{aligned}
\]
Now we take into account that for any $v\in T_p\mathsf{S}(V)$ we have 
$(v\mathbf{i})_{\mathcal{V}^\perp}=(v_{\mathcal{V}^\perp})\mathbf{i}$, since $\mathcal{V}^\perp$ is $\H$-invariant, to get
\[
\begin{aligned}
(\nabla_{X_p}J_1)Y
=\nabla_{X_p}(J_1 Y)-J_1(\nabla_{X_p}Y)
=\pi_{*\sigma(p)}\bigl(Y_{\sigma(p)}^L[\lambda'(0),\mathbf{i}]\bigr).
\end{aligned}
\]

Recall from the definition of $\lambda$ that 
$X_{\sigma(p)}^L=\tilde{\gamma}'(0)=\sigma_{*p}(X_p)+\sigma(p)\lambda'(0)$.
Thus, we define, taking into account that $X_{\sigma(p)}^L$ is horizontal,
\[
\begin{aligned}
q_1(X)&{}=\frac{1}{2}\langle\sigma_{*p}(X_p),\sigma(p)\mathbf{i}\rangle
=-\frac{1}{2}\langle \sigma(p)\lambda'(0),\sigma(p)\mathbf{i}\rangle
=-\frac{1}{2}\langle \lambda'(0),\mathbf{i}\rangle,\\
q_2(X)&{}=\frac{1}{2}\langle\sigma_{*p}(X_p),\sigma(p)\mathbf{j}\rangle
=-\frac{1}{2}\langle \sigma(p)\lambda'(0),\sigma(p)\mathbf{j}\rangle
=-\frac{1}{2}\langle \lambda'(0),\mathbf{j}\rangle,\\
q_3(X)&{}=\frac{1}{2}\langle\sigma_{*p}(X_p),\sigma(p)\mathbf{k}\rangle
=-\frac{1}{2}\langle \sigma(p)\lambda'(0),\sigma(p)\mathbf{k}\rangle
=-\frac{1}{2}\langle \lambda'(0),\mathbf{k}\rangle,
\end{aligned}
\]
where the last inner product is the standard inner product of $\R^3\equiv\Im\H$.

The bracket relations of the Lie algebra $\g{sp}(1)=\Im\H$ imply
\[
[\lambda'(0),\mathbf{i}]
=2\langle\lambda'(0),\mathbf{k}\rangle\,\mathbf{j}-2\langle\lambda'(0),\mathbf{j}\rangle\,\mathbf{k}.
\]
This yields,
\[
\begin{aligned}
(\nabla_{X_p}J_1)Y
&{}=\pi_{*\sigma(p)}\bigl(Y_{\sigma(p)}^L(2\langle\lambda'(0),\mathbf{k}\rangle\,\mathbf{j}-2\langle\lambda'(0),\mathbf{j}\rangle\,\mathbf{k})\bigr)
=-q_3(X)J_2 Y+q_2(X)J_3 Y.
\end{aligned}
\]

With a similar argument we can get $\nabla_X J_i=-q_{i+2}(X)J_{i+1}+q_{i+1}(X)J_{i+2}$, indices modulo~3, from where the result follows.
\end{proof}

Finally we show

\begin{proposition}
$\mathsf{P}(V)$ has constant quaternionic sectional curvature $4$.
\end{proposition}

\begin{proof}
Let $p\in\mathsf{P}(V)$.
We have to show that any $2$-plane in a quaternionic line of $T_p\mathsf{P}(V)$ has the same sectional curvature.
As usual we denote orthogonal projection onto $\mathcal{V}=\ker\pi_*$ by a subindex.

Let $N$ be the unit normal vector field of the sphere $N_x=x$, for $x\in\mathsf{S}(V)$.
First we calculate, for $X$, $Y\in\Gamma(T\mathsf{P}(V))$ and $\lambda\in\Sp(1)$,
\[
\begin{aligned}
\langle \tilde{\nabla}_{X^L} Y^L,N\lambda\rangle
=-\langle Y^L,D_{X^L}(N\lambda)\rangle
=-\langle Y^L,(D_{X^L}N)\lambda\rangle
=-\langle Y^L,X^L \lambda\rangle
=\langle X^L,Y^L \lambda\rangle.
\end{aligned}
\]

Let $\g{q}$ be a local quaternionic structure as defined above, and $J\in\g{q}$ a complex structure.
Let $X\in T_p\mathsf{P}(V)$ with $\lvert X\rvert=1$. 
We have to calculate the sectional curvature of the $2$-plane spanned by $X$ and $JX$, and see that this does not depend on $p$ or $X$.

First we have 
\[
\langle[X^L,(JX)^L],N\lambda\rangle
=\langle X^L,(JX)^L\lambda\rangle-\langle(JX)^L,X^L\lambda\rangle
=2\langle X^L,(JX)^L\lambda\rangle.
\]
This implies $\lvert[X^L,(JX)^L]_\mathcal{V}\rvert^2=4\lvert X^L\rvert^2$.
Using~\cite[Corollary~1]{ONeill}, the formula above, and since the sphere of radius $1$ has sectional curvature $1$, we get
\[
K(X, JX)
=1+\frac{3}{4}\bigl\lvert[X^L,JX^L]_\mathcal{V}\bigr\rvert^2
=4,
\]
as we wanted to show.
\end{proof}

In $V$ we can consider the action of $\Sp(V)\cong\Sp(n+1)$ on the left as
$A\cdot\mathbf{x}=A\mathbf{x}$, where matrix multiplication is $\H$-linear on the right.
This action leaves the sphere $\mathsf{S}(V)$ invariant because
\[
\langle A\mathbf{x},A\mathbf{x}\rangle
=\Re(A\mathbf{x},A\mathbf{x})
=\Re(\mathbf{x},\mathbf{x})
=\langle \mathbf{x},\mathbf{x}\rangle=1.
\]
Moreover $\pi(A(\mathbf{x}\lambda))=\pi((A\mathbf{x})\lambda)=\pi(A\mathbf{x})$, so the action descends to $\mathsf{P}(V)$ as $A\cdot\pi(\mathbf{x})=\pi(A\cdot\mathbf{x})$.

The action of $\Sp(V)$ on $\mathsf{P}(V)$ is transitive because $\Sp(V)$ acts transitively on quaternionic lines through the origin.
We show that it is also isometric.
Let $p\in\mathsf{P}(V)$ and $v$, $w\in T_p\mathsf{P}(V)$.
Let $\alpha\colon I\to\mathsf{P}(V)$ be a smooth curve such that $\alpha(0)=p$, $\alpha'(0)=v$.
We select $x\in\pi^{-1}(p)$ and let $\tilde{\alpha}$ be the horizontal lift of $\alpha$ to $\mathsf{S}(V)$ through $x$.
Then, $\tilde{\alpha}(0)=x$ and $\tilde{\alpha}'(0)=v_x^L$.
Since $A$ acts linearly on $V$,
\[
A_{*p}(v)
=\frac{d}{dt}\Big\vert_0 A(\alpha(t))
=\frac{d}{dt}\Big\vert_0 A(\pi(\tilde{\alpha}(t)))
=\frac{d}{dt}\Big\vert_0 \pi(A\tilde{\alpha}(t))
=\pi_{*x}(A\tilde{\alpha}'(0))
=\pi_{*x}(Av_x^L).
\]
Now, since $\pi$ is a Riemannian submersion
\[
\langle A_{*p}(v),A_{*p}w\rangle
=\langle \pi_{*x}(A v_x^L),\pi_{*x}(A w_x^L)\rangle
=\langle A v_x^L,A w_x^L\rangle
=\langle v_x^L,w_x^L\rangle
=\langle v,w\rangle.
\]

The effectivity kernel of $\Sp(V)$ is $\mathbb{Z}_2=\{\pm \id\}$.
Indeed, if $A\in\Sp(V)$ acts as the identity on $\mathsf{P}(V)$, for each $x\in\mathsf{S}(V)$ we have 
$\pi(Ax)=A\pi(x)=\pi(x)$.
Hence, for each $x\in\mathsf{S}(V)$ there exists $\lambda(x)\in\Sp(1)$ such that $Ax=x\lambda(x)$.
It is not hard to see that this implies that $A$ is diagonal;
but since the action of $A$ is on the left and the quaternions that commute with any other element are reals numbers, we get that $A$ is a diagonal matrix with real entries.
Thus, $A=\pm\id$.

Let $\{e_1,\dots,e_{n+1}\}$ be a quaternionic orthonormal basis of $V$.
From now on we fix $o=\pi(e_1)$.
We calculate the isotropy group of $\Sp(V)$ at $o$.
If $A(\pi(e_1))=A(o)=o$ then there exists $\lambda\in\Sp(1)$ such that $Ae_1=e_1\lambda$.
Since $A\in\Sp(V)$ we get $A=\lambda\oplus B$ where $\lambda\in\Sp(1)$, $B\in\Sp(W)$, and where $W$ is the quaternionic orthogonal complement of $e_1$ in $V$.
Therefore the isotropy group of $\Sp(V)$ at $o$ is isomorphic to $\Sp(1)\times\Sp(n)$.

Now we calculate the isotropy representation $\Sp(V)_o\times T_o\mathsf{P}(V)\to T_o\mathsf{P}(V)$.
We identify $\Sp(V)_o\cong \Sp(1)\times\Sp(n)$ as before, and $T_o\mathsf{P}(V)\cong(e_1\H)^\perp\cong\H^{n}$.
If $A=(\lambda,B)\in\Sp(1)\times\Sp(n)$, and $v\in\H^n$ we get
\[
\begin{aligned}
A_{*o}(v)
&{}=\pi_{*e_1}(Av_{e_1}^L)
=\frac{d}{dt}\Big\vert_0 \pi(A(e_1\cos t+v_{e_1}^L\sin t))\\
&{}=\frac{d}{dt}\Big\vert_0 \pi(e_1(\cos t) \lambda+B v_{e_1}^L\sin t)\\
&{}=\frac{d}{dt}\Big\vert_0 \pi(e_1(\cos t)+B v_{e_1}^L\lambda^{-1}\sin t)\\[1ex]
&{}=\pi_{*e_1}(Bv_{e_1}^L\lambda^{-1}).
\end{aligned}
\]
Moreover, the effectivity kernel of this action is $\mathbb{Z}_2=\{\pm\id\}$ (similar argument to the action of $\Sp(V)$ on $\mathsf{P}(V)$).
Thus, the isotropy representation of $\mathsf{P}(V)$ is equivalent to the representation
\[
\begin{array}{r@{\ }c@{\ }l}
\Sp(1)\Sp(n)\times\H^n & \to & \H^n\\
((\lambda,B),v) & \mapsto & Bv\lambda^{-1},
\end{array}
\]
where $\Sp(1)\Sp(n)=(\Sp(1)\times\Sp(n))/\mathbb{Z}_2$.


\section{Quaternionic hyperbolic spaces}

The construction of quaternionic hyperbolic spaces is in many ways similar to the projective case.
In this section we outline this construction and leave out the details whenever they are analogous to the previous section.
We focus on specific properties of the hyperbolic case, though.

Let $V$ be a quaternion vector space of quaternionic dimension $n+1$ endowed with an indefinite quaternion product $(\,\cdot\,,\,\cdot\,\,)$ of signature $(1,n)$.
This means that $V$ is isomorphic to $\H^{1,n}$, that is, the quaternion vector space $\H^{n+1}$ endow with the indefinite quaternion product 
\[
(\mathbf{x},\mathbf{y})
=-\overline{x}_1{y}_1+\sum_{i=2}^{n+1}\overline{x}_i {y}_i.
\]
This implies that $\langle\,\cdot\,,\,\cdot\,\,\rangle=\Re(\,\cdot\,,\,\cdot\,\,)$ is an scalar product of signature $(4,n)$.
As a Riemannian manifold, $V$ is isometric to the flat pseudo-Riemannian space $\R^{4,4n}$

We consider the $\mathsf{S}^{3,4n}=\mathsf{S}(V)=\{{x}\in V:\langle {x},{x}\rangle=-1\}$,
which is a pseudo-Riemannian manifold of signature $(3,4n)$ and constant negative curvature.
We define the proper and free right action $\mathsf{S}(V)\times\Sp(1)\to\mathsf{S}(V)$ by ${x}\cdot \lambda={x}\lambda$.
The orbit space 
\[
\mathsf{H}(V)=\mathsf{S}(V)/\Sp(1)
\]
is a smooth manifold of real dimension $4n$, that is called the quaternion hyperbolic space of $V$.
If $V=\H^{1,n}$, we denote this space by $\H \mathsf{H}^n$.

We denote by $\pi\colon\mathsf{S}(V)\to\mathsf{H}(V)$ the quotient map, also called the Hopf map.
For ${x}\in\mathsf{S}(V)$ we identify $T_{x}\mathsf{S}(V)=(\R{x})^\perp=V\ominus\R{x}$, the orthogonal complement of ${x}$ with respect to $\langle\,\cdot\,,\,\cdot\,\,\rangle$.
We have $\ker\pi_{*{x}}={x}\Im\H$, and the orthogonal decomposition $T_{x}\mathsf{S}(V)=({x}\Im\H)\oplus(\H{x})^\perp$.
We endow $\mathsf{H}(V)$ with a Riemannian metric that makes $\pi$ a pseudo-Riemannian submersion.
Thus, $T_{\pi({x})}\mathsf{P}(V)$ becomes isometric to $(\H{x})^\perp$ through $\pi_{*{x}}$.
As before, a vector in $(\H x)^\perp$ is called horizontal, and a vector in $\ker\pi_{*x}$ is called vertical.

Let $v\in T_{p}\mathsf{P}(V)$.
For ${x}\in\pi^{-1}(p)$ we define the horizontal lift $v^L_{x}$ of $v$ as the unique vector $v_x^L\in(\H{x})^\perp$ such that $\pi_{*{x}}(v^L_{x})=v$.
The metric on $\mathsf{H}(V)$ is defined as 
$\langle v,w\rangle_{\mathsf{H}(V)}=\langle v^L,w^L\rangle_{\mathsf{S}(V)}$ 
for each $v$, $w\in T_p\mathsf{H}(V)$.
This definition does not depend on the point ${x}\in\pi^{-1}(p)$ used to define the lift.

Since $\pi\colon\mathsf{S}(V)\to\mathsf{H}(V)$ is a Riemannian submersion, the Levi-Civita connection $\nabla$ of $\mathsf{P}(V)$ is determined by
$(\nabla_X Y)_{\pi(x)}=\pi_{*x}(\tilde{\nabla}_X Y)$,
where $\tilde{\nabla}$ is the Levi-Civita connection of $\mathsf{S}(V)$.

The quaternion K\"ahler structure on $\mathsf{P}(V)$ is defined as follows.
Let $\sigma\colon U\to\mathsf{S}(V)$ be a section of $\pi$.
We define, for $X\in\Gamma(T\mathsf{H}(V))$,
\[
\begin{aligned}
J_1(X_p)&{}=\pi_{*\sigma(p)}(-X_{\sigma(p)}^L\mathbf{i}),
&J_2(X_p)&{}=\pi_{*\sigma(p)}(-X_{\sigma(p)}^L\mathbf{j}),
&J_3(X_p)&{}=\pi_{*\sigma(p)}(-X_{\sigma(p)}^L\mathbf{k}),
\end{aligned}
\]
for $p\in U$.
This definition depends on the choice of section $\sigma$, but the span of $\{J_1,J_2,J_3\}$ determines a quaternionic K\"ahler structure on $\mathsf{H}(V)$.

A calculation similar to the projective case shows that $\mathsf{H}(V)$ has constant quaternionic sectional curvature $-4$. 
Note that the normal vector to $\mathsf{S}(V)$ is now timelike and that $S(V)$ has constant curvature $-1$.
\medskip

In $V$ we consider the action of $\Sp(V)\cong\Sp(1,n)$ on the left as
$A\cdot\mathbf{x}=A\mathbf{x}$, where matrix multiplication is $\H$-linear on the right.
This action leaves $\mathsf{S}(V)$ invariant and descends to $\mathsf{P}(V)$ as $A\cdot\pi(\mathbf{x})=\pi(A\cdot\mathbf{x})$.
The action of $\Sp(V)$ on $\mathsf{H}(V)$ is transitive, because $\Sp(V)$ acts transitively on quaternionic lines through the origin, and isometric because
$A_{*p}(v)=\pi_{*x}(Av_x^L)$ for any $p\in\mathsf{H}(V)$, $x\in\pi^{-1}(p)$, and $v\in T_p\mathsf{H}(V)$, and $\pi$ is a pseudo-Riemannian submersion.
The effectivity kernel of $\Sp(V)$ is again $\mathbb{Z}_2=\{\pm \id\}$.

Let $\{e_1,\dots,e_{n+1}\}$ be a quaternionic orthonormal basis of $V$ with $(e_1,e_1)=-1$ and $(e_i,e_i)=1$ if $i>1$.
From now on we fix $o=\pi(e_1)$.
The isotropy group of $\Sp(V)$ at $o$ is isomorphic to $\Sp(1)\times\Sp(n)$, 
and the isotropy representation $\Sp(V)_o\times T_o\mathsf{H}(V)\to T_o\mathsf{H}(V)$
is equivalent to the representation
\[
\begin{array}{r@{\ }c@{\ }l}
\Sp(1)\Sp(n)\times\H^n & \to & \H^n\\
((\lambda,B),v) & \mapsto & Bv\bar{\lambda}.
\end{array}
\]

\begin{proposition}
Let $\{e_1,\dots,e_n\}$ be the canonical basis of $\H^n$, and $\{J_1,J_2,J_3\}$ the canonical basis for the quaternionic structure of $\H^n$.
Let $L\colon\H^n\to\H^n$ be an $\R$-linear map.
Then, $L\in\Sp(1)\Sp(n)$ if and only if $L$ maps $\{J_1,J_2,J_3\}$ to a canonical basis of the quaternionic structure of $\H^n$ and $L$ maps $\{e_1,\dots,e_n\}$ to an $\H$-orthonormal basis of $\H^n$.
\end{proposition}

\begin{proof}
Let $\{J_1',J_2',J_3'\}$ be a canonical basis of the quaternionic structure of $\H^n$, and $\{e_1',\dots,e_n'\}$ an $\H$-orthonormal basis of $\H^n$.
Assume that $Le_i=e_i'$, $i\in\{1,\dots,n\}$, and $L\circ J_i=J_i'\circ L$, $i\in\{1,2,3\}$.
We show that $L\in\Sp(1)\Sp(n)$.

Conversely, assume $L\in\Sp(1)\Sp(n)$, and let $e_i'=Le_i$ and $J_i'=L\circ J_i\circ L^{-1}$.
We have to show that $\{J_1',J_2',J_3'\}$ is a canonical basis of the quaternionic structure of $\H^n$, and $\{e_1',\dots,e_n'\}$ is an $\H$-orthogonal basis of $\H^n$.

TO DO
\end{proof}

We consider the map $\phi_o\colon \Sp(1,n)\to\mathsf{H}(V)$, $A\mapsto A(o)$.
We have its differential $\phi_{o*}\colon\g{sp}(1,n)\to T_o\mathsf{H}(V)$, given by
\[
\phi_{*o}(X)
=\frac{d}{dt}\Big\vert_{0}\phi(\Exp(tX))
=\frac{d}{dt}\Big\vert_{0}e^{tX}(\pi(e_1))
=\frac{d}{dt}\Big\vert_{0}\pi(e^{tX}e_1)
=\pi_{*e_1}(Xe_1).
\]
Clearly, its kernel is the Lie algebra $\g{sp}(1)\oplus\g{sp}(n)$ of the isotropy group of $\Sp(V)$ at $o$.
Thus, complementary space of $\g{sp}(1)\oplus\g{sp}(n)$ in $\g{sp}(1,n)$ is isomorphic to $T_o\mathsf{H}(V)$, which is itself isomorphic to $\H^n$.
We will see that there are two natural choices for this complementary subspace.

The Lie algebra $\g{g}=\g{sp}(1,n)$ can be viewed as the algebra of quaternionic matrices of the following form:
\[
\g{sp}(1,n)=\left\{
\left(
\begin{array}{c|c}
\lambda & v^{*} \\
\hline
v & X
\end{array}
\right):
\lambda \in \Im \H,v\in \H^{n}, B\in \g{sp}(n)
\right\}.
\]
We use the following notation: 
\[
\lceil \lambda, v, X\rceil =\left(
\begin{array}{c|c}
\lambda & v^{*} \\
\hline
v & X
\end{array}
\right).
\]
We have
\[
\bigl[\lceil \lambda,v,X \rceil,\, \lceil \mu,w,Y \rceil\bigr]
=\lceil \lambda\mu-\mu\lambda+v^{*}w-w^{*}v,\, v\mu-w\lambda+Xw-Yv,\, [X,Y]+vw^{*}-wv^{*} \rceil.
\]

The Cartan decomposition $\g{g}=\g{k}\oplus\g{p}$ is given by
\[
\begin{aligned}
\g{k}&
{}=\g{sp}(1)\oplus\g{sp}(n)
=\left\{
	\left(
	\begin{array}{c|c}
		\lambda & 0 \\
		\hline
		0 & X
	\end{array}
	\right):
	\lambda \in \Im \H, B\in \g{sp}(n)
\right\}, \\
\g{p}&
{}=\left\{
\left(
	\begin{array}{c|c}
		0 & v^{*} \\
		\hline
		v & 0
	\end{array}
\right):
v\in \H ^{n}
\right\}.
\end{aligned}
\]
The Cartan involution of $\g{g}$ is $\theta X = -X^{*}$.

Since $\ker\phi_{*o}=\g{k}$, $\phi_{*o}\colon\g{p}\to T_o\mathsf{H}(V)$ is a vector space isomorphism.
We endow $\g{p}$ with the inner product and quaternion structure induced by $\phi_{*o}$.
This is clearly the canonical identification $\lceil 0,v,0\rceil\mapsto \pi_{*e_1}(v)$.

\begin{remark}
	The trace of a quaternionic matrix $A\in \g{gl}(n,\H)$ is defined as $\tr_{\C}(A)=2 \Re\tr_{\H} A)$ 
	(that is, $\tr_{\C}(A)$ is the trace of $A$ regarded as a complex linear operator, see Section~\ref{sec:vector-spaces}).
\end{remark}

The Killing form of $\g{sp}(1,n)$ is given by $\mathcal{B}(X,Y)=2\pdfmarkupcomment{(n+1)}{Check this constant}\tr_{\C}(XY)$.
Indeed, the complexification of $\g{g}$ is $\g{sp}(n+1,\C)$, whose Killing form is precisely the one given above (extended in the obvious way).
This induces an inner product on $\g{g}$ given by 
\[
\mathcal{B}_{\theta}(X,Y)
=-\mathcal{B}(\theta X, Y)
=4(n+1)\Re\tr_{\H}(X^{*}Y).
\]

We have 
\[
\mathcal{B}_\theta(\lceil 0,v,0\rceil,\lceil 0,w,0\rceil)
=4(n+1)\Re\tr_\H(\lceil v^*w,0,vw^*\rceil)
=8(n+1)\langle v,w\rangle.
\]
Thus, we rescale $\mathcal{B}_\theta$ so that $\phi_{*o}\colon\g{p}\to T_o\mathsf{H}(V)$ is an isometry, and set
\[
\langle X,Y\rangle
=\frac{1}{8(n+1)}\mathcal{B}_\theta(X,Y)
=\frac{1}{2}\Re\tr_\H(X^*Y),
\]
for $X$, $Y\in\g{g}$.

We choose the maximal abelian subspace $\g{a}$ of $\g{p}$ generated by $\lceil 0,e_{1},0\rceil$.
The root space decomposition $\g{g}=\g{g}_{-2\alpha}\oplus\g{g}_{-\alpha}\oplus\g{g}_{0}\oplus\g{g}_{\alpha}\oplus\g{g}_{2\alpha}$ is given by
\[
\begin{aligned}
\g{g}_{0}={}&\left\{
	\left(
		\begin{array}{cc|c}
			\lambda & t & 0 \\
			t & \lambda & 0 \\
			\hline
			0 & 0 & X
		\end{array}
	\right)\colon \lambda\in\Im\H,t\in\R,X\in\g{sp}(n-1)
\right\}=\g{k}_{0}\oplus \g{a}, \\
\g{k}_{0}={}&\left\{
\left(
\begin{array}{cc|c}
	\lambda & 0 & 0 \\
	0 & \lambda & 0 \\
	\hline
	0 & 0 & X
\end{array}
\right)\colon \lambda\in\Im\H,X\in\g{sp}(n-1)
\right\}=\g{sp}(1)\oplus \g{sp}(n-1), \\
\g{g}_{\alpha}={}&\left\{
	\left(
		\begin{array}{cc|c}
			0 & 0 & v^{*} \\
			0 & 0 & v^{*} \\
			\hline
			v & -v & 0
		\end{array}
	\right)\colon v\in \H^{n-1}
\right\}, \\
\g{g}_{2\alpha}={}&\left\{
	\left(
		\begin{array}{cc|c}
			\lambda & -\lambda & 0 \\
			\lambda & -\lambda & 0 \\
			\hline
			0 & 0 & 0
		\end{array}
	\right)\colon \lambda\in \Im \H
\right\},
\end{aligned}
\]
where $\alpha\in\g{a}^*$ is given by $\alpha(\lceil 0,e_1,0\rceil)=1$, 
and $\g{g}_{-\alpha}=\theta\g{g}_\alpha$, $\g{g}_{-2\alpha}=\theta\g{g}_{2\alpha}$.
Let ${K}_{0}$ be the connected subgroup of ${G}=\Sp(1,n)$ with Lie algebra $\g{k}_{0}$.
Then we get
\[
{K}_{0}=\left\{
	\left(
		\begin{array}{cc|c}
			q & 0 & 0 \\
			0 & q & 0 \\
			\hline
			0 & 0 & B
		\end{array}
	\right)\colon q\in\mathsf{Sp}(1),B\in\mathsf{Sp}(n-1)
\right\}
\cong\mathsf{Sp}(1)\times\mathsf{Sp}(n-1).
\]
	Furthermore, the (effectivized) adjoint representation $\mathsf{K}_{0}\curvearrowright\g{g}_{\alpha}\cong\H^{n-1}$ is precisely the action of $\mathsf{Sp}(1)\mathsf{Sp}(n-1)$ on $\H^{n-1}$ via the identification $v\leftrightarrow\lceil 0,v,0 \rceil$, whereas the adjoint representation $\mathsf{K}_{0}\curvearrowright\g{g}_{2\alpha}$ satisfies $(q,B)\cdot \lambda=q\lambda\bar{q}$.
	

--------------------


	We have the natural identification $\g{p}\cong T_{o}\H \mathsf{H}^{n}=\H e_{1}^{\perp}$ given by $\lceil 0,v,0 \rceil\leftrightarrow (0,v)$.
	In particular, we may consider the endomorphisms $Q_{\lambda}\in \mathsf{End}(T_{o}\H\mathsf{H}^{n})$ (where $\lambda\in \H$) defined as follows:
	\begin{equation}
		Q_{\lambda}\xi=\pi_{*e_{1}}\left(\xi^{L}_{e_{1}}\lambda\right).
	\end{equation}
	Here, $\xi^{L}_{e_{1}}\in \{0\}\times \H^{n}$ is the horizontal lift of $\xi\in T_{o}\H\mathsf{H}^{n}$ to $e_{1}$.
	One checks that $\mathfrak{J}=\{Q_{\lambda}\colon \lambda\in \operatorname{Im}\H\}$ defines a quaternionic structure on $T_{o}\H\mathsf{H}^{n}$.
	In particular, we can translate this quaternionic structure to $\g{p}$, which actually satisfies
	\begin{equation*}
		Q_{\lambda}\lceil 0,v,0\rceil=\lceil 0,v\lambda,0 \rceil.
	\end{equation*}
	Let $\mathsf{A}$, $\mathsf{N}$ and $\mathsf{AN}$ be the connected subgroups of $\mathsf{Sp}(1)\times\mathsf{Sp}(n-1)$ associated with $\g{a}$, $\g{n}$ and $\g{a}\oplus \g{n}$.
	Then it is a standard fact that $\H\mathsf{H}^{n}$ is a principal homogeneous space of $\mathsf{AN}$, so the map $g\in\mathsf{AN}\to g\cdot o\in\H\mathsf{H}^{n}$ is a diffeomorphism.
	We consider the metric $\langle\cdot,\cdot \rangle_{\mathsf{AN}}$ that turns this map into an isometry.
	Explicitly,
	\begin{equation}
		\langle X,Y \rangle_{\mathsf{AN}}=\left\langle \frac{1-\theta}{2}X,\frac{1-\theta}{2}Y \right\rangle=\langle X_{\g{a}},Y_{\g{a}} \rangle+\frac{1}{2}\langle X_{\g{n}},Y_{\g{n}}\rangle.
	\end{equation}
	In particular, the quaternionic structure on $\g{p}$ can be transformed into a quaternionic structure on $\g{a}\oplus\g{n}$ by making $\frac{1-\theta}{2}$ into a quaternionic vector space isomorphism.
	Explicitly:
	\begin{equation}
		Q_{\lambda}\left(
			\begin{array}{cc|c}
				\mu & t-\mu & v^{*} \\
				t+\mu & -\mu & v^{*} \\
				\hline
				v & -v & 0
			\end{array}
		\right)=
		\left(
		\begin{array}{cc|c}
			\operatorname{Im}((t+\mu)\lambda) & \bar{\lambda}(t-\mu) & \bar{\lambda}v^{*} \\
			(t+\mu)\lambda & -\operatorname{Im}((t+\mu)\lambda) & \bar{\lambda}v^{*} \\
			\hline
			v\lambda & -v\lambda & 0
		\end{array}
		\right)		
	\end{equation}
	
	To every element $X\in\g{g}_{2\alpha}$, we associate a skew-symmetric map $J_{X}\colon\g{g}_{\alpha}\to\g{g}_{\alpha}$ as follows:
	\begin{equation*}
		X=\left(
		\begin{array}{cc|c}
			\lambda & -\lambda & 0 \\
			\lambda & -\lambda & 0 \\
			\hline
			0 & 0 & 0
		\end{array}
		\right),
		V=\left(
			\begin{array}{cc|c}
				0 & 0 & v^{*} \\
				0 & 0 & v^{*} \\
				\hline
				v & -v & 0
			\end{array}
		\right)
		\Rightarrow
		J_{X}V=2\left(
			\begin{array}{cc|c}
				0 & 0 & -\lambda v^{*} \\
				0 & 0 & -\lambda v^{*} \\
				\hline
				v\lambda & -v\lambda & 0
			\end{array}
		\right).
	\end{equation*}
	
	Define vectors $B$, $X_{1}$, $X_{2}$, $X_{3}\in\g{a}\oplus \g{n}$ as
	\begin{equation}
		B=\left(
			\begin{array}{cc|c}
				0 & 1 & 0 \\
				1 & 0 & 0 \\
				\hline
				0 & 0 & 0
			\end{array}
		\right),
		X_{1}=\left(
		\begin{array}{cc|c}
			i & -i & 0 \\
			i & -i & 0 \\
			\hline
			0 & 0 & 0
		\end{array}
		\right),
		X_{2}=\left(
		\begin{array}{cc|c}
			j & -j & 0 \\
			j & -j & 0 \\
			\hline
			0 & 0 & 0
		\end{array}
		\right),X_{3}=\left(
		\begin{array}{cc|c}
			k & -k & 0 \\
			k & -k & 0 \\
			\hline
			0 & 0 & 0
		\end{array}
		\right),
	\end{equation}
	and define $J_{X_{1}}$, $J_{X_{2}}$ and $J_{X_{3}}$ (in $\g{a}\oplus \g{g}_{2\alpha}$) as the endomorphisms of $\g{a}\oplus \g{g}_{2\alpha}$ corresponding to right multiplication by $i$, $j$ and $k$ respectively under the identification $\g{a}\oplus \g{g}_{2\alpha}\to \H e_{2}\subseteq T_{o}\H\mathsf{H}^{n}$.
	In particular, we have
	\begin{equation*}
		\begin{split}
			J_{X_{i}}B={}&X_{i},\quad J_{X_{i}}X_{i}=-B,\quad J_{X_{i}}X_{i+1}=X_{i+2},\quad J_{X_{i}}X_{i+2}=-J_{X_{i+1}},
		\end{split}
	\end{equation*}
	where the indices in the last two equations are taken modulo $3$.
	
	\textcolor{red}{\textbf{WARNING!!!!!} The definition of $J_{X}$ given by making $\g{a}\oplus\g{n}$ isomorphic to $T_{o}\H \mathsf{H}^{n}$ does not agree with the $J_{X}\in\mathsf{End}(\g{g}_{\alpha})$ given above.}
	
	This is trivially a complex structure if and only if $\lvert \lambda \rvert^{2}=\frac{1}{4}$ (that is, $\lvert X \rvert^{2}=2\lvert X \rvert^{2}_{\mathsf{AN}}=2$), and more generally, $J_{X}^{2}=-\frac{\lvert \lambda \rvert^{2}}{4}\operatorname{Id}_{\g{g}_{\alpha}}$,
	We also have \textcolor{blue}{$\langle[U,V],X \rangle=\langle J_{X}U,V\rangle$}.
	All in all, we conclude that $(\g{n}=\g{g}_{\alpha}\oplus \g{g}_{2\alpha},\langle \cdot ,\cdot \rangle_{\mathsf{AN}})$ becomes a generalized Heisenberg algebra with Clifford map $J\colon X\in \g{g}_{2\alpha}\mapsto J_{X}\in\mathsf{End}(\g{g}_{\alpha})$.
	The following formulae are easy to check:
	let $\mathfrak{J}\subseteq \mathsf{End}_{\R}(\g{g}_{\alpha})$ be the quaternionic structure induced by the maps $J_{X}$, $X\in \g{g}_{2\alpha}$.
	Then the Lie bracket $\g{g}_{\alpha}\times\g{g}_{\alpha}\to \g{g}_{2\alpha}$ satisfies
	\begin{equation}
		\begin{split}
			[U,V]={}&0 \Leftrightarrow V\in \R U\oplus (\g{g}_{\alpha}\ominus \mathfrak{J}U), \\
			[J_{X}U,U]={}&
		\end{split}
	\end{equation} 
	

\begin{thebibliography}{99}
\bibitem{ONeill}
B.~O'Neill: The fundamental equations of a submersion, \textit{Michigan Math.\ J.}\ \textbf{13} (1966), 459--469.
\end{thebibliography}
\end{document}